 \documentclass[12pt]{article}
 %File creation date: 10/23/18  at 11:40
 \usepackage{pstricks,pst-node,pst-tree}
 \usepackage{geometry}
 \usepackage{lscape}
 \pagestyle{empty}
 \begin{document}
 %\begin{landscape}
 \begin{center}
\psset{linecolor=black,tnsep=1pt,tndepth=0cm,tnheight=0cm,treesep=1.2cm,levelsep=56pt,radius=10pt}
  \pstree[treemode=D]{\Tcircle{ \normalsize 1 }~[tnpos=l]{\shortstack[r]{\texttt{\detokenize{V1}}\\$\leq$20.09}}
    ~[tnpos=r]{\shortstack[r]{0.68\\0.32}}}{
 \Tcircle[fillcolor=green,fillstyle=solid]{ 2 }
    ~[tnpos=l]{\shortstack[r]{0.41\\0.59}}
    ~{\shortstack{\makebox[0pt][c]{\texttt{\detokenize{Normal}}}\\165}}
 \Tcircle[fillcolor=yellow,fillstyle=solid]{ 3 }
    ~[tnpos=r]{\shortstack[r]{0.99\\0.01}}
    ~{\shortstack{\makebox[0pt][c]{\texttt{\detokenize{Abnormal}}}\\145}}
 }
 \end{center}
GUIDE v.29.0  0.50-SE pruned
classification tree for predicting \texttt{\detokenize{classification}} using
equal priors
and unit misclassification costs.
Maximum number of split levels is 10 and minimum node sample size is 2.
At each split, an observation goes to the left branch 
 if and only if the condition is satisfied.
\texttt{\detokenize{V1}} = \texttt{\detokenize{degree_spondylolisthesis}}.
Predicted classes and sample sizes printed below terminal nodes;
 class proportions for \texttt{\detokenize{classification}} =
 \texttt{\detokenize{Abnormal}} and \texttt{\detokenize{Normal}} beside nodes.
Second best split variable at root node is \texttt{\detokenize{pelvic_radius}}.
 %\end{landscape}
 \end{document}
